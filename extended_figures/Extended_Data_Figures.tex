\documentclass{article}

% https://www.nature.com/nmeth/submission-guidelines/aip-and-formatting
% Figure panels should be prepared at a minimum resolution of 300 dpi and
% saved at a maximum width of 180 mm (7").
% Use a 5–7 pt san serif font for standard text labelling and Symbol font for
% Greek characters.

\usepackage[scaled]{helvet} % Use Helvetica, a sans-serif font
\usepackage{graphicx}
\usepackage[top=0.75in, bottom=0.75in, left=0.75in, right=0.75in]{geometry}
\usepackage{caption}
\usepackage{xcolor}
\usepackage{underscore}

% Set the default font to sans-serif
\renewcommand{\familydefault}{\sfdefault}

% Customize caption settings
\captionsetup{
  font={sf},
  textfont={},
  labelformat=empty,
  labelsep=none
}

\def\nocaptiontrue{}
\newif\ifnocaption
\ifdefined\nocaptiontrue
  \nocaptiontrue
\else
  \nocaptionfalse
\fi

\renewcommand\captionfont{\fontsize{8pt}{10pt}\selectfont}

% with size: \inclimgsz{pdf}{width}{label}{caption}
\newcommand{\inclimgsz}[4]{
\begin{figure}
  \centering
  \includegraphics[width=#2, keepaspectratio]{#1}
  \unless\ifnocaption
    \caption{\textbf{#3.} {#4}}
  \fi
\end{figure}
\clearpage
\newpage
}

% basic: \inclimg{pdf}{label}{caption}
\newcommand{\inclimg}[3]{\inclimgsz{#1}{180mm}{#2}{#3}}

% smaller: \inclimgsm{pdf}{label}{caption}
\newcommand{\inclimgsm}[3]{\inclimgsz{#1}{150mm}{#2}{#3}}

% \inclcropsz{pdf}{bottom trim}{width}{label}{caption}
\newcommand{\inclcropsz}[5]{
\begin{figure}
  \centering
  \includegraphics[width=#3, trim={0 #2 0 0}, clip, keepaspectratio]{#1}
  \unless\ifnocaption
     \caption{\textbf{#4.} {#5}}
   \fi
\end{figure}
\clearpage
\newpage
}

% \inclcrop{pdf}{bottom trim}{label}{caption}
\newcommand{\inclcrop}[4]{\inclcropsz{#1}{#2}{180mm}{#3}{#4}}

% generate images: \inclgen{pdf}{label}{caption}
\newcommand{\inclgen}[3]{\inclcrop{#1}{25pt}{#2}{#3}}

% generate images: \inclgen{pdf}{label}{caption}
\newcommand{\inclgenb}[3]{\inclcrop{#1}{20pt}{#2}{#3}}

% \warning{text}
\newcommand{\warning}[1]{\textcolor{red}{\textbf{#1}}}

% \notice{text}
\newcommand{\notice}[1]{\textcolor{green}{\textbf{#1}}}

% \unused{text}
\newcommand{\unused}[1]{\textcolor{red}{\textbf{***UNUSED*** #1}}}


\begin{document}

%%%%
%% challenge 1
%%%%
%KEEP!!
\inclimg{challenge1_src/images/summary-values.pdf}{Extended Data Fig. 0.5.}{
  \warning{Lorem ipsum dolor sit amet, consectetur adipiscing
      elit, sed do eiusmod tempor incididunt ut labore et dolore magna
      aliqua. Ut enim ad minim veniam, quis nostrud exercitation ullamco
      laboris nisi ut aliquip ex ea commodo consequat.}
}
\inclimg{challenge1_src/images/read-usage-by-tool.pdf}{Extended Data Fig. 1.}{
  Read usage by analysis tool. a-c) The Percentage of Reads
  Used (PRU) is calculated as the fraction between the number of reads in
  transcript models provided in the submission of each pipelines and the
  number of available reads in the dataset. Values > 100 indicate the same
  read is assigned to more than one transcript model. Values < 100 indicate
  that not all available reads were used to predict transcript models. d)
  Distribution of the number of transcripts assigned to each long-read in the
  submitted reads2transcripts files. Values are aggregated for all submissions
  of the same tool.
}
\inclimg{challenge1_src/images/sqanti3-eval-h1-mix.pdf}{Extended Data Fig. 2.}{
  SQANTI3 evaluation of LRGASP submissions of the H1-mix dataset. Labels
  correspond to analysis tools and the color code indicates the combination of
  library preparation and sequencing platform. a) Number of gene and
  transcript detections. b) Number of Full Splice Match and Incomplete Splice
  Match transcripts. c) Number of Novel in Catalogue and Novel Not in
  Catalogue transcripts. d) Number of known and novel transcripts with full
  support at junctions and end positions. e) Percentage of transcripts with
  5´end support. f) Percentage of transcripts with 3´end support. g)
  Percentage of canonical splice junctions (SJ) and short-reads support at SJ.
  Ba: Bambu, FM: Flames, FL: FLAIR, IQ: IsoQuant, IT: IsoTools, IB: Iso_IB,
  Ly: LyRic, Ma: Mandalorion, TL: TALON-LAPA, Sp: Spectra, ST: StringTie2.
}
\inclimg{challenge1_src/images/sqanti3-eval-es.pdf}{Extended Data Fig. 3.}{
  SQANTI3 evaluation of LRGASP submissions of the mouse ES dataset. Labels
  correspond to analysis tools and the color code indicates the combination of
  library preparation and sequencing platform. a) Number of gene and
  transcript detections. b) Number of Full Splice Match and Incomplete Splice
  Match transcripts. c) Number of Novel in Catalogue and Novel Not in
  Catalogue transcripts. d) Number of known and novel transcripts with full
  support at junctions and end positions. e) Percentage of transcripts with
  5´end support. f) Percentage of transcripts with 3´end support. g)
  Percentage of canonical splice junctions (SJ) and short-reads support at SJ.
  Ba: Bambu, FM: Flames, FL: FLAIR, IQ: IsoQuant, IT: IsoTools, IB: Iso_IB,
  Ly: LyRic, Ma: Mandalorion, TL: TALON-LAPA, Sp: Spectra, ST: StringTie2.
}
\inclgen{challenge1_src/generated/Extended_Fig._4_seq-depth-by-detected-features.pdf}{Extended Data Fig. 4.}{
  Relationship between sequencing depth and number of detected features. a-c) Transcripts, d-f)
  Genes.
}
\inclgen{challenge1_src/generated/Extended_Fig._5_read-length-by-detected-features.pdf}{Extended Data Fig. 5.}{
  Relationship between read length and number of detected features. a-c) Transcripts, d-f) Genes.
}
\inclgen{challenge1_src/generated/Extended_Fig._6_read-qual-by-detected-features.pdf}{Extended Data Fig. 6.}{
  Relationship between read quality and number of detected features. a-c) Transcripts, d-f) Genes.
}
\inclgen{challenge1_src/generated/Extended_Fig._7_deviance-dectected-features-by-expr-factors.pdf}{Extended Data Fig. 7.}{
  Median Absolute Deviance of detected features by experimental factor. a-c) Transcripts, d-f) Genes.
}
\inclgen{challenge1_src/generated/Extended_Fig._8_detected-trans-gene-by-tool.pdf}{Extended Data Fig. 8.}{
  Number of detected transcripts and genes per analysis tool. a-c) Transcripts, d-f) Genes.
}
\inclgen{challenge1_src/generated/Extended_Fig._9_detected-genes-by-plat-prep.pdf}{Extended Data Fig. 9.}{
  Number of detected genes per Platform and Library Preparation. a-c) Platform, d-f) Library
  Preparation.
}
\inclgen{challenge1_src/generated/Extended_Fig._10_detected-genes-by-plat-prep.pdf}{Extended Data Fig. 10.}{
  Number of detected transcripts per Platform and Library Preparation.
}
\inclgen{challenge1_src/generated/Extended_Fig._11_detected-trans-by-cDNA-CapTrap.pdf}{Extended Data Fig. 11.}{
  Number of
  detected transcripts in cDNA and CapTrap libraries. a-c) cDNA, d-f) CapTrap.
}
\inclgen{challenge1_src/generated/Extended_Fig._12_detected-trans-by-PacBio-ONT.pdf}{Extended Data Fig. 12.}{
  Number of
  detected transcripts in PacBio and Nanopore platforms. a-c) PacBio, d-f) Nanopore.
}
\inclgen{challenge1_src/generated/Extended_Fig._13_detected-genes-by-PacBio-ONT.pdf}{Extended Data Fig. 13.}{
  Number of
  detected genes in cDNA and CapTrap libraries. a-c) cDNA, d-f) CapTrap.
}
\inclgen{challenge1_src/generated/Extended_Fig._14_detected-genes-by-PacBio-ONT.pdf}{Extended Data Fig. 14.}{
  Number of
  detected genes in PacBio and Nanopore platforms. a-c) PacBio, d-f) Nanopore.
}
\inclgen{challenge1_src/generated/Extended_Fig._15_FSM-ISM-by-plat-lib.pdf}{Extended Data Fig. 15.}{
  Number of FSM
  and ISM by sequencing platform and library preparation. a-c) FSM, d-f) ISM.
}
\inclgen{challenge1_src/generated/Extended_Fig._16_NIC-NCC-by-plat-prep.pdf}{Extended Data Fig. 16.}{
  Number of NIC
  and NNC by sequencing platform and library preparation. a-c) NIC, d-f) NNC.
}
\inclgen{challenge1_src/generated/Extended_Fig._17_FSM-by-lib-tool.pdf}{Extended Data Fig. 17.}{
  Number of FSM
  transcripts by library preparation and analysis tool. a-c) cDNA. d-f) CapTrap.
}
\inclgen{challenge1_src/generated/Extended_Fig._18_FSM-by-plat-tool.pdf}{Extended Data Fig. 18.}{
  Number of FSM
  transcripts by sequencing platform and analysis tool. a-c) PacBio, d-f) Nanopore.
}
\inclgen{challenge1_src/generated/Extended_Fig._19_ISM-by-prep-tool.pdf}{Extended Data Fig. 19.}{
  Number of ISM
  transcripts by library preparation and analysis tool. a-c) cDNA. d-f) CapTrap.
}
\inclgen{challenge1_src/generated/Extended_Fig._20_ISM-by-plat-tool.pdf}{Extended Data Fig. 20.}{
  Number of ISM
  transcripts by sequencing platform and analysis tool. a-c) Intergenic. d-f) GenicGenomic.
}
\inclgen{challenge1_src/generated/Extended_Fig._21_genic-by-plat-prep.pdf}{Extended Data Fig. 21.}{
  Number of
  Intergenic and GenicGenomic by sequencing platform and library preparation. a-c) Intergenic,
  d-f) GenicGenomic.
}
\inclgen{challenge1_src/generated/Extended_Fig._22_fusion-anti-by-plat-prep.pdf}{Extended Data Fig. 22.}{
  Number of Fusion and Antisense by sequencing platform and library preparation. a-c)
  Fusion. d-f) Antisense.
}
%KEEP!!
\inclcrop{challenge1_src/generated/Extended_Fig._23_TMs-coverage-by-tool.pdf}{38pt}{Extended Data Fig. 23.}{
  Percentage of transcript models (TM) with different ranges of sequence
  coverage by long reads. a) WTC11. c) H1-mix. c) Mouse ES. Ba: Bambu, FM: FLAMES, FL: FLAIR,
  IQ: IsoQuant, IT: IsoTools, IB: Iso_IB, Ly: LyRic, Ma: Mandalorion, TL: TALON-LAPA, Sp:
  Spectra, ST: StringTie2.
}
\inclgen{challenge1_src/generated/Extended_Fig._24_biotype-by-tool.pdf}{Extended Data Fig. 24.}{
  Distribution of Biotypes across pipelines. a) WTC11, c) H1-mix, c) Mouse ES.
}
\inclgen{challenge1_src/generated/Extended_Fig._25_biotype-per-pipeline.pdf}{Extended Data Fig. 25.}{
  Biotypes per
  pipeline. a) WTC11, c) H1-mix, c) Mouse ES.
}
\inclgen{challenge1_src/generated/Extended_Fig._26_cnt_squanti_dist-UIC.pdf}{Extended Data Fig. 26.}{
  Number and
  SQANTI category distribution of Unique Intron Chain (UIC) consistently detected by an
  increasing number of submissions. a) H1-mix sample, b) Mouse ES sample.
}
\inclimg{challenge1_src/images/detection-overlap-wtc11.pdf}{Extended Data Fig. 27.}{
  Pair-wise overlap in the detection of features between pipelines; WTC11
  sample. Each value represents the feature intersection between column and
  row pipelines divided by the number of detections in the row pipeline. a)
  Genes, b) Splice junctions, c) Unique Intron Chains (UIC), c) Top UIC
  accounting for at least 50\% of the gene expression.
  \warning{UPDATE: Image is fuzzy}
}
\inclimg{challenge1_src/images/detection-overlap-h1-mix.pdf}{Extended Data Fig. 28.}{
  Pair-wise overlap in the detection of features between pipelines; H1-mix
  sample. Each value represents the feature intersection between column and
  row pipelines divided by the number of detections in the row pipeline. a)
  Genes, b) Splice junctions, c) Unique Intron Chains (UIC), c) Top UIC
  accounting for at least 50\% of the gene expression.
  \warning{UPDATE: Image is fuzzy}
}
\inclimg{challenge1_src/images/detection-overlap-es.pdf}{Extended Data Fig. 29.}{
  Pair-wise overlap in the detection of features between pipelines; ES mouse
  sample. Each value represents the feature intersection between column and
  row pipelines divided by the number of detections in the row pipeline. a)
  Genes, b) Splice junctions, c) Unique Intron Chains (UIC), c) Top UIC
  accounting for at least 50\% of the gene expression.
  \warning{UPDATE: Image is fuzzy}
}
\inclgen{challenge1_src/generated/Extended_Fig._30_UIC-by-tool-PacBio_cDNA.pdf}{Extended Data Fig. 30.}{
  Number of UIC detected by a tool and shared with an increasing number of other tools,
  processing PacBio_cDNA data. a) WTC11, c) H1-mix, c) Mouse ES.
}
\inclgen{challenge1_src/generated/Extended_Fig._31_UIC-by-tool-PacBio_CapTrap.pdf}{Extended Data Fig. 31.}{
  Number of UIC detected by a tool and shared with an increasing number of other tools,
  processing PacBio_CapTrap data. a) WTC11, c) H1-mix, c) Mouse ES.
}
\inclgen{challenge1_src/generated/Extended_Fig._32_UIC-by-tool-ONT_cDNA.pdf}{Extended Data Fig. 32.}{
  Number of UIC
  detected by a tool and shared with an increasing number of other tools, processing
  ONT_cDNA data. a) WTC11, c) H1-mix, c) Mouse ES.
}
\inclgen{challenge1_src/generated/Extended_Fig._33_UIC-by-tool-ONT_CapTrap.pdf}{Extended Data Fig. 33.}{
  Number of UIC
  detected by a tool and shared with an increasing number of other tools, processing
  ONT_CapTrap data. a) WTC11, c) H1-mix, c) Mouse ES.
}
\inclgen{challenge1_src/generated/Extended_Fig._34_UIC-by-tool-ONT_R2C2.pdf}{Extended Data Fig. 34.}{
  Number of UIC
  detected by a tool and shared with an increasing number of other tools, processing ONT_R2C2
  data. a) WTC11, c) H1-mix, c) Mouse ES
}
\inclgen{challenge1_src/generated/Extended_Fig._35_UIC-by-tool-ONT_dRNA.pdf}{Extended Data Fig. 35.}{
  Number of UIC
  detected by a tool and shared with an increasing number of other tools, processing ONT_dRNA
  data. a) WTC11, c) H1-mix, c) Mouse ES
}
\inclgen{challenge1_src/generated/Extended_Fig._36_UIC-by-tool-sample.pdf}{Extended Data Fig. 36.}{
  Number of UIC
  consistently detected by a tool across samples. a) WTC11, c) H1-mix, c) Mouse ES
}
\inclimg{challenge1_src/images/frequent-uic-characterization.pdf}{Extended Data Fig. 37.}{
  Characterization of frequently detected UICs (FDU).  a,c,e) Structural
  category distribution of FDU. The table indicates the fold enrichment of
  each structural category within the frequently detected transcripts respect
  to their global count. b,d,f) Tools identifying FDU. The graph shows the
  enrichment in the number FDU found by a tool with respect to their global
  number of reported transcripts. The table reports the total number of FDU
  detected by the tool.
}
\inclgen{challenge1_src/generated/Extended_Fig._38_trans-prop-by-prep.pdf}{Extended Data Fig. 38.}{
  Properties of
  detected transcripts by library preparation. a,d,g) Length distribution. b,e,h) Exon number
  distribution. c,f,i) Counts per million
}
\inclgen{challenge1_src/generated/Extended_Fig._39_trans-prop-by-plat.pdf}{Extended Data Fig. 39.}{
  Properties of
  detected transcripts by platform. a,d,g) Length distribution. b,e,h) Exon number
  distribution. c,f,i) Counts per million
}
\inclgen{challenge1_src/generated/Extended_Fig._40_trans-by-protocol.pdf}{Extended Data Fig. 40.}{
  Properties of
  detected transcripts by experimental protocol. a,d,g) Length distribution. b,e,h) Exon number
  distribution. c,f,i) Counts per million
}
\inclgen{challenge1_src/generated/Extended_Fig._41_trans-len-by-tool.pdf}{Extended Data Fig. 41.}{
  Distribution of
  transcript length by analysis tool.
}
%KEEP!!
\inclimg{challenge1_src/images/sirv-cover-cDNA_PacBio.pdf}{Extended Data Fig. 42a.}{
  Positional coverage of SIRV transcript sequences by long reads
  in the cDNA_PacBio sample.
}
\inclimg{challenge1_src/images/sirv-cover-CapTrap_PacBio.pdf}{Extended Data Fig. 42b.}{
  Positional coverage of SIRV transcript sequences by long reads
  in the CapTrap_PacBio.
}
%KEEP!!
\inclimg{challenge1_src/images/sirv-cover-cDNA_ONT.pdf}{Extended Data Fig. 42c.}{
  Positional coverage of SIRV transcript sequences by long reads
  in the cDNA_ONT sample.
}
\inclimg{challenge1_src/images/sirv-cover-CapTrap_ONT.pdf}{Extended Data Fig. 42d.}{
  Positional coverage of SIRV transcript sequences by long reads
  in the CapTrap_ONT.
}
\inclimg{challenge1_src/images/sirv-cover-R2C2_ONT.pdf}{Extended Data Fig. 42e.}{
  Positional coverage of SIRV transcript sequences by long reads
  in the R2C2_ONT sample.
}
\inclimg{challenge1_src/images/sirv-cover-dRNA_ONT.pdf}{Extended Data Fig. 42f.}{
  Positional coverage of SIRV transcript sequences by long reads
  in the dRNA_ONT sample.
}
\inclcrop{challenge1_src/generated/Extended_Fig._43_metrics-mouse-simul.pdf}{68pt}{Extended Data Fig. 43.}{
  Performance
  metrics on mouse simulated data. Sen_kn: sensitivity known transcripts, Sen_kn > 5TMP:
  sensitivity known transcripts with expression > 5 TPM, Pre_kn: precision known transcripts,
  Sen_no: sensitivity novel transcripts, Pre_no: precision novel transcripts, 1/Red: inverse of
  redundancy.}
\inclgen{challenge1_src/generated/Extended_Fig._44_trans-cover-real-simul.pdf}{Extended Data Fig. 44.}{
  Comparison of long-read transcript coverage between real and simulated datasets.
}
%KEEP!!
\inclcrop{challenge1_src/generated/Extended_Fig._45_gencode-manual-props-wtc11.pdf}{38pt}{Extended Data Fig. 45.}{
  Properties of
  GENCODE manually annotated loci for WTC11 sample.a) Distributon of gene expression. b)
  Distribution of SQANTI categories. c) Intersection of Unique Intron Chains (UIC) among
  experimental protocols.
}
%KEEP!!
\inclcrop{challenge1_src/generated/Extended_Fig._46_gencode-manual-props-es.pdf}{38pt}{Extended Data Fig. 46.}{
  Properties of GENCODE manually annotated loci for mouse ES sample.a)
  Distributon of gene expression. b) Distribution of SQANTI categories. c) Intersection of
  Unique Intron Chains (UIC) among experimental protocols.
}
\inclcrop{challenge1_src/generated/Extended_Fig._47_pipeline-by-gencode-manual.pdf}{38pt}{Extended Data Fig. 47.}{
  Performance metrics of LRGASP pipelines evaluate against GENCODE manual annotation of mouse
  ES sample. Ba: Bambu, FM: Flames, FR: FLAIR, IQ: IsoQuant, IT: IsoTools, IB: Iso_IB, Ly:
  LyRic, Ma: Mandalorion, TL: TALON-LAPA, Sp: Spectra, ST: StringTie2.
}
\inclcrop{challenge1_src/generated/Extended_Fig._48_uic-gencode-manual-by-tool.pdf}{38pt}{Extended Data Fig. 48.}{
  Detection of Unique Intron Chains (UIC) at GENCODE manual annotation loci. Ba: Bambu, FM:
  Flames, FL: FLAIR, IQ: IsoQuant, IT: IsoTools, IB: Iso_IB, Ly: LyRic, Ma: Mandalorion, TL:
  TALON-LAPA, Sp: Spectra, ST: StringTie2.
}
\inclcrop{challenge1_src/generated/Extended_Fig._49_gencode-manual-by-tool-two-datasets.pdf}{38pt}{Extended Data Fig. 49.}{
  Performance on GENCODE manually curated data. Curated transcripts selected to be present in
  at least two experimental datasets. Ba: Bambu, FM: Flames, FL: FLAIR, IQ: IsoQuant, IT:
  IsoTools, IB: Iso_IB, Ly: LyRic, Ma: Mandalorion, TL: TALON-LAPA, Sp: Spectra, ST:
  StringTie2.
}
\inclcrop{challenge1_src/generated/Extended_Fig._50_gencode-manual-by-tool-three-reads.pdf}{38pt}{Extended Data Fig. 50.}{
  Performance on GENCODE manually curated data. The ground truth is the set of manually
  annotated transcripts with more than two reads. Ba: Bambu, FM: Flames, FL: FLAIR, IQ:
  IsoQuant, IT: IsoTools, IB: Iso_IB, Ly: LyRic, Ma: Mandalorion, TL: TALON-LAPA, Sp: Spectra,
  ST: StringTie2.
}
\inclgen{challenge1_src/generated/Extended_Fig._51_gencode-manual-by-prep.pdf}{Extended Data Fig. 51.}{
 Performance on GENCODE manually curated data by Library Preparation.
}
\inclgen{challenge1_src/generated/Extended_Fig._52_gencode-manual-by-plat.pdf}{Extended Data Fig. 52.}{
  Performance on GENCODE manually curated data by Platform.
}
%%%%
%% challenge 2
%%%%
\inclimg{challenge2_src/fig_rador_plot.pdf}{Extended Data Fig. 1_overall_Radar_plot.}{
  Radar plot of overall evaluation results of eight quantification tools with
  seven protocols-platforms on four data scenarios: real data with multiple
  replicates, cell mixing experiment, SIRV-set 4 data and simulation data. To
  display the evaluation results more effectively, we normalized all metrics
  to 0-1 range: 0 corresponds to the worst performance and 1 corresponds to
  the best performance.
}
\inclimg{challenge2_src/fig_ranking_on_six_protocols_and_platforms.pdf}{Extended Data Fig. 2_Ranking_results.}{
  Top 3 performance on quantification tools under six different
  protocols-platforms for each metric. Here, quantification tools showcase
  scores under six different protocols-platforms across various evaluation
  metrics, with the top 3 performers highlighted for each metric. Blank spaces
  denote instances where the tool or protocols-platforms did not have
  participants submitting the corresponding quantitative results.
}
%%%%
%% challenge 3
%%%%
\inclimg{challenge3_src/Extended_Fig._63.pdf}{Extended Data Fig. 63.}{
  Manatee genome assembly statistics. a Nanopore reads were used to obtain a
  draft genome of the Floridian manatee with Flye. The resulting assembly was
  polished with existing Illumina reads using Pilon. b BUSCO completeness.
}
\inclimg{challenge3_src/Extended_Fig._64.pdf}{Extended Data Fig. 64.}{
  Mapping rate of transcript detected by Challenge 3 submissions.
}
\inclimg{challenge3_src/Extended_Fig._65.pdf}{Extended Data Fig. 65.}{
  SQANTI category classification of transcript models detected by the same
  tools in Challenge 1 and Challenge 1 predictions used the reference
  annotation and Challenge 3 predictions did not. Ba = Bambu, IQ =
  StringTie2/IsoQuant.
}
\inclimg{challenge3_src/Extended_Fig._66.pdf}{Extended Data Fig. 66.}{
  Coding potential of transcripts detected by Challenge 3 submissions.
}
\inclimg{challenge3_src/Extended_Fig._67.pdf}{Extended Data Fig. 67.}{
  SQANTI3 analysis of SIRV reads in manatee samples. a) SQANTI3 categories for
  reads mapping to SIRVs in cDNA-PacBio and cDNA-ONT replicates. b) Number of
  SIRV transcripts with at least one Reference Match (RM) read in cDNA-PacBio
  and cDNA-ONT replicates.
}
%%%%
%% validation
%%%%
\inclimg{validation_src/Extended_Data_Fig68_validation_by_supportive_reads.pdf}{Extended Data Fig. 68.}{
  Fraction of validated WTC-11 transcripts as a function of the total numbers
  of long reads that were observed across the 21 library preparations (e.g.,
  PacBio cDNA, ONT cDNA, PacBio CapTrap).
}
\inclimg{validation_src/Extended_Data_Fig69_transcript_lengths_by_validation_status.pdf}{Extended Data Fig. 69.}{
  The distribution of lengths corresponding to the target transcript isoform
  across the entire validation experiment (including GENCODE, Platform, and
  Consistency groups), broken down by their validation status.
}
\inclimg{validation_src/Extended_Figure_70.pdf}{Extended Data Fig. 70.}{
  PCR validation results for manatee isoforms for seven target genes (data
  shown in Figure 5l) broken down by the platform (ONT or PacBio) underlying
  the pipelines that led to the identification of the isoform.
}
\inclimg{validation_src/Extended_Figure_71.pdf}{Extended Data Fig. 71.}{
  Validation of ALG6 U12 Intron with WTC11 Reads. In panel (a), a novel
  transcript model, NCC_39352 (blue arrow), appears to corroborate the exon
  within the ALG6 GENCODE annotation. The mapped amplicon in the control
  junction tracks provides evidence of the preceding intron. The green arrow
  indicates the ONT and PacBio read alignment coverage over the exon, but the
  junction tracks shows a lack of support for the splice junction at the
  exon's 5' end. In panel (b), GENCODE's annotation of a rare U12 GT-AT intron
  (purple arrow), which is unsupported by minimap2. Instead, minimap2 forces a
  GT-AG intron by reporting a six-base deletion in the reference genome (red
  arrow). As all pipelines relied on minimap2, correct annotation of this
  transcript was unattainable, illustrating the challenges difficult-to-align
  regions can pose to annotation with long- read transcripts.
}
\end{document}
